\documentclass{article}
\usepackage{graphicx}
\usepackage{geometry}
\usepackage{color}
\usepackage{booktabs}
\usepackage{ listings} 



 \geometry{
 a4paper,
 total={170mm,257mm},
 left = 30mm,
 right = 30mm,
 top=20mm,
 }
\begin{document}

\begin{figure}
\centering
\includegraphics[width = \hsize]{welcomePage.png}
\end{figure}

\title{ 
History of Presidential Elections \\ 
CMSC Database Design Project\\
}
\author{
Dong Yuan, Weiru Xie\\
University of Maryland
}
\date{}


\maketitle


\newpage
\tableofcontents
\newpage

%\includegraphics{welcomePage.png}

\section{ENVIRONMENT AND REQUIREMENT ANALYSIS}
\subsection{Purpose of Document}
The purpose of this document is to provide detailed requirements and design specifications as well as to describe the implementation process and result for the CMSC424 Database Design Project \textbf{History of Presidential Elections}. The document contains a description of limitations and assumptions about this model, a description of how to extract, transform and load data and how to build a web server. A top-level flow diagram is included in this document to show the logical flow of this project. Besides that, there are document forms as well as task forms to describe different tasks at each stage. For conceptual modeling, there is an ER model graphic schema and some functional dependencies derived. For logical modeling, there is a logical schema of the relational model. The document also has a list of design specification of each tasks, a user manual about each query that this project implements and some additional information.

\subsection{Purpose of Project}
The main goal of this project is to populate a relational database, called the Presidential Elections Database (PED), with data readily available on the Web. In this project, at ETL part, we need transform web HTML and XML data into a relational database that supports SQL querying and processing, format the data and integrate the data into a single relational database. Then we need to create a web-interface for users to interact and query the database. At last, a detailed report to describe the analysis, design and implementation process will be produced.

\subsection{Scope}
The scope of this project involves multiple tasks. The first task is to research and collect reliable data source related to the history of the president elections in United States(1789 to 2016). The second is extract data from these web sites and to do some "data cleaning" during transformation and loading. That is to discover and eliminate duplicate data, correct wrong data and transfer data into uniform format. The third task is to build a welcome page and a select query page. It provides an interface for users to create a query to search for information on a given election year, information on a given president/candidate, presidents who were re-elected on non-contiguous times, swing candidates, results of a party throughout the history of the elections, information on a given state, and so on. This task also needs us to write code to process and interpret these mentioned queries and provide results to users.

\subsection{Assumption}
The assumptions for this project are as follows:
\begin{itemize}
	\item The data will be accurate, reliable and complete.
	\item The user has basic web browsing skills to access the web interface.
	\item It is assumed that the database server is configured appropriately to handle
the user demands placed on the project.
\end{itemize}

\subsection{Technical and Conceptual Problems and Solutions}
\subsubsection{Technical Problems and Solutions}
{\bfseries Problem:} Implementing a web server\\
{\bfseries Solutions:} Research to acquire necessary knowledge to install and run a web server on a demo machine.\\

\noindent{\bfseries Problem:} Gathering data\\
{\bfseries Solutions:} Search online and compare data on different websites. \\

\noindent{\bfseries Problem:} Extracting data\\
{\bfseries Solutions:} For useful and reliable data, we choose different ways to extract data considering their format. Write a web crawler in Python to extract data from Wekipedia.\\

\noindent{\bfseries Problem:} Transforming data\\
{\bfseries Solutions:} Transform extracted data into uniform format.\\

\noindent{\bfseries Problem:} Loading data\\
{\bfseries Solutions:} Load the reformatted data into our database.\\

\noindent{\bfseries Problem:} Lack of knowledge in creating interactive web pages\\
{\bfseries Solutions:} Research and learn languages like HTML, CSS, PHP. \\

\noindent{\bfseries Problem:} Lack of knowledge creating web server scripts\\
{\bfseries Solutions:} Research and learn JSP. \\

\noindent{\bfseries Problem:} Writing accurate and detailed pseudocode for each task and the embedded DML code \\
{\bfseries Solutions:} Further research the technologies and starting the programming phase. \\

\noindent{\bfseries Problem:} Building the SQL queries \\
{\bfseries Solutions:} Research and follow examples. \\

\noindent{\bfseries Problem:} Returning results and display to the client side\\
{\bfseries Solutions:} Research and follow examples. \\

{\color{red} (May add things in later phrase)}


\subsubsection{Conceptual Problems and Solutions}
{\bfseries Problem:} Identifying a complete set of tasks of the project.\\
{\bfseries Solutions:} Deeper analysis of this project. Look through the sample of a complete project and learn from it. Talk to the teaching assistant for advice. \\

\noindent{\bfseries Problem:} Designing additional queries.\\
{\bfseries Solutions:} Analyze our extracted data and their relationship to com up with our own queries. Ask teaching assistant for approval.\\

\noindent{\bfseries Problem:} Designing the flow chat.\\
{\bfseries Solutions:} Deeper analysis and understanding of this project. Figure out what we should do at each phrase of project.\\

\noindent{\bfseries Problem:} Designing and building the E-R model.\\
{\bfseries Solutions:} Review concepts of E-R model. Analyze extracted data and their relationship. Build a E-R model with less redundancy and more convenience for queries.\\


\section{SYSTEM ANALYSIS AND SPECIFICATION}
\subsection{Description of Procedure}
PED operates via a web browser that allows users to select various search criteria in researching facts about American presidential election from 1789 through 2016.
\subsubsection{From the User's Perspective}
The first step in accessing XXX(the name of our website) is to navigate a predefined website. There the user will create a query and submit it to a process running on a remote server where the PED is stored. This process will create a form containing SQL commands for the specified query and will submit it to the database. After the data has been retrieved from the database it will be formatted and presented through the user?s web browser.\\
\includegraphics[height=250pt]{eps3.eps}
\subsubsection{From the Developer's Perspective}
The developers will be employing several technologies in order to implement the enterprise in its varying phases. The diagram below shows the main components of the system and indicates what responsibility to the system each component has as well as the general flow of information. Parts of this diagram will be elaborated upon in subsequent sections.\\
\includegraphics[width = \hsize]{png2.png}
\subsubsection{ETL Process}
The designers research, analyze, and select the most relevant presidential election information websites. With the resulting websites bookmarked, Python scripts are used to extract useful data from the resulting tables, transform it into the required format. !!!! What is used???? to load data into the OlympicsDB tables located on the Oracle server to be used by OlympiChronicles to answer the different user queries through a web interface.\\
\includegraphics[width = \hsize]{png1.png}

\subsubsection{Web server Procedures}
The XXX internal procedures include the engine that powers the website. This involves providing various queries and results of those queries available to the user. This is accomplished by several scripts and code for pages located on a UNIX web server. When a user navigates to the XXX website, the initial web page is generated and served to the user. The remaining scripts and procedures will be described by following a typical use case scenario. The user will proceed to enter the website. A page with a list of nine queries is presented to the user. The user will then select one from among the queries to be performed. The selection will be sent to the web server where another procedure will generate a page with various options relevant to that query. The user will fill out the options desired and submit the query. The web server will receive the query request; generate the appropriate SQL commands which are then sent to the XXX. The database will then produce a results table and send it to the server from which the query was sent. Another process on the server will format the results into a web page and serve it up to the user.\\
\includegraphics[height=300pt]{eps1.eps}


\subsection{Documentation}


\subsubsection{Top-Level Flow Diagram}
\includegraphics[height=450pt]{eps2.eps}

\subsubsection{Tasks, Subtasks, and Task Forms}

\begin{table}[h!]
  \centering
  \caption{Pages Research Task.}
  \label{tab:table5}
  \begin{tabular}{lp{10cm}}
       \hline
     TASK NAME & Web Pages Research\\
     PERFORMER & PED designers\\
     PURPOSE & To research the internet for web sites that contain data for the American presidential election from 1789 to 2016. \\
     DESCRIPTION & Research the internet.\\
     ENABLING COND & To populate the PED.\\
     FREQUENCY & As often as necessary\\
     DURATION & Varies\\
     INPUT & Web queries\\
     OUTPUT & Index of queried results\\
     DOCUMENT USE &Web based search engines \\
     OPS PERFORMED & Researching and bookmarking web sites and/or pages with American presidential election data \\
     SUBTASKS & None\\
     ERROR COND &  None\\
       \hline
  \end{tabular}
\end{table}

\begin{table}[h!]
  \centering
  \caption{ETL Task.}
  \label{tab:table5}
  \begin{tabular}{lp{10cm}}
       \hline
     TASK NAME & Extract, Transform, and Load Task \\
     PERFORMER & {\color{red}TODO}\\
     PURPOSE & To extract data, transform or reformat it and load it into the PED \\
     DESCRIPTION & This tool {\color{red}To be decided} extracts specific data from a web page, and load it into a predefined data relation or table.\\
     ENABLING COND & The creation of the OlympicsDB and any addition of data or updates to the OlympicsDB\\
     FREQUENCY & Once for the creation of the OlympicsDB and during any updates. \\
     DURATION & Varies \\
     INPUT & A selected web page\\
     OUTPUT & Data into a relation in the PED\\
     DOCUMENT USE & HTML documents\\
     OPS PERFORMED & Data extraction, data transformation, and data loading\\
     SUBTASKS & Web pages Research \\
     ERROR COND &  None\\
       \hline
  \end{tabular}
\end{table}

\begin{table}[h!]
  \centering
  \caption{Generate Welcome Page Task.}
  \label{tab:table5}
  \begin{tabular}{lp{10cm}}
       \hline
     TASK NAME & Generate Welcome Page.\\
     PERFORMER & Apache web server\\
     PURPOSE & To generate the welcome page.\\
     DESCRIPTION & The Apache/Tomcat web server will generate the XXX(name of our database website) welcome page (home page) when a user wants to access the information on the PED. \\
     ENABLING COND & User accessing the XXX web interface. \\
     FREQUENCY & As often as a user accesses the web address \\
     DURATION & Very short\\
     INPUT & None\\
     OUTPUT & Welcome Page\\
     DOCUMENT USE & WIFWF: Web Interface Welcome Form\\
     OPS PERFORMED & Generation of welcome page, send it to the user and wait for user action\\
     SUBTASKS & None\\
     ERROR COND &  If A/TServer == busy, then Process=TimeOut. \\
       \hline
  \end{tabular}
\end{table}

\begin{table}[h!]
  \centering
  \caption{Generate Query Select Page Task.}
  \label{tab:table5}
  \begin{tabular}{lp{10cm}}
       \hline
     TASK NAME & Generate Query Select Page.\\
     PERFORMER & Apache web server\\
     PURPOSE & To generate the query select page.\\
     DESCRIPTION & The Apache/Tomcat web server will generate the XXX query-select page when requested by the user (from the Welcome page) \\
     ENABLING COND & User clicking on the ENTER button in the Welcome page. \\
     FREQUENCY & As often as the user clicks on the ENTER button on the Welcome Page or the BACK button on a Query Page. \\
     DURATION & Very short\\
     INPUT & Signal request from user to web server. \\
     OUTPUT & Query Select Page\\
     DOCUMENT USE & WISF: Web Interface Select Form\\
     OPS PERFORMED & Generate the Query Select page, send it to the user and wait for user input. \\
     SUBTASKS & None\\
     ERROR COND &  If A/TServer == busy, then Process=TimeOut. \\
       \hline
  \end{tabular}
\end{table}


\begin{table}[h!]
  \centering
  \caption{Generate Query Page Task.}
  \label{tab:table5}
  \begin{tabular}{lp{10cm}}
       \hline
     TASK NAME & Generate Query Page\\
     PERFORMER & Apache web server\\
     PURPOSE & To build a query page for users to select their options for a specific query\\
     DESCRIPTION & Every time when a user clicks on a query to make their selections, the Apache web server will generate a query page.\\
     ENABLING COND & Clicking on a query on the Query Select Page\\
     FREQUENCY & As often as a user clicks on a query\\
     DURATION & Very short\\
     INPUT & Request from user to web server\\
     OUTPUT & Query page\\
     DOCUMENT USE & WIQF: Web Interface Query Form\\
     OPS PERFORMED & Provide query form to users and  allow users to make their selections. When they submit, the server will receive the input.\\
     SUBTASKS & None\\
     ERROR COND &  If server is busy, then Process == TimeOut\\
       \hline
  \end{tabular}
\end{table}

\begin{table}[h!]
  \centering
  \caption{Generate SQL Query Task}
  \label{tab:table6}
  \begin{tabular}{lp{10cm}}
       \hline
     TASK NAME & Generate SQL Query\\
     PERFORMER & Apache web server\\
     PURPOSE & To create SQL query based on users input.\\
     DESCRIPTION & Every time when a user submits a web query form, corresponding SQL commands will be generated to task the presidential elections database.\\
     ENABLING COND & Submitting a web query form.\\
     FREQUENCY & As often as a user submits a web query form.\\
     DURATION & Short\\
     INPUT & Web query form\\
     OUTPUT & SQL form\\
     DOCUMENT USE & {\color{red}TODO}\\
     OPS PERFORMED &  {\color{red}TODO}\\
     SUBTASKS & None\\
     ERROR COND &  If server is busy, then Process == TimeOut\\
       \hline
  \end{tabular}
\end{table}

\begin{table}[h!]
  \centering
  \caption{Create Query Results Form Task}
  \label{tab:table7}
  \begin{tabular}{lp{10cm}}
       \hline
     TASK NAME & Create Query Results Form\\
     PERFORMER & Server side script\\
     PURPOSE & To provide a formatted result from the Presidential Elections DB.\\
     DESCRIPTION & To transform the result of a success query in PED into a format that can be interpreted by a web browser.\\
     ENABLING COND & Database completing operations.\\
     FREQUENCY & As often as a user submits a web query form and the query is performed successfully.\\
     DURATION & Depends on the complexity of the query result.\\
     INPUT & Data in Presidential Elections DB.\\
     OUTPUT &  {\color{red}TODO}\\
     DOCUMENT USE & None\\
     OPS PERFORMED &  Transform the result of a success query in PED into a format that can be interpreted by a web browser.\\
     SUBTASKS & None\\
     ERROR COND &  If the output is unknown, then produce error message and stop.\\
       \hline
  \end{tabular}
\end{table}


\begin{table}[h!]
  \centering
  \caption{Generate Results Page Task}
  \label{tab:table8}
  \begin{tabular}{lp{10cm}}
       \hline
     TASK NAME & Generate Results Page\\
     PERFORMER & Apache web server\\
     PURPOSE & To generate results page.\\
     DESCRIPTION & Every time when the query on Presidential Elections DB is performed, a query result form will be generated. The server will generate the results page and display it to users.\\
     ENABLING COND & Getting the query result form.\\
     FREQUENCY & As often as a user submits a web query form and the query is performed successfully.\\
     DURATION & Short\\
     INPUT & Query result form\\
     OUTPUT & Results page\\
     DOCUMENT USE & WIRF: Web Interface Result Form\\
     OPS PERFORMED &  Generate a Results Page to be displayed to the user from the Result form.\\
     SUBTASKS & None\\
     ERROR COND &  If server is busy, then Process == TimeOut\\
       \hline
  \end{tabular}
\end{table}

\begin{tabular}{|c|c|}
        \hline
        Column 1 & Column 2\\
        \hline
        second row & \\
        third row & \\
        \hline
\end{tabular}


\subsubsection{Document Forms}
\includegraphics[height=\hsize]{png3.png}

\section {Task Emulation}
\subsection{Task Design Specification}
\subsubsection{Extract, Transform and Load Task Design}
\emph{Google query to find American Presidents Elections sites}\\
\begin{lstlisting}
if website has complete and reliable data to be used by the ElectionsDB 
    Bookmark
else 
    skip
\end{lstlisting}

\noindent \emph{ Write web crawlers in Python}
\begin{lstlisting}
for each website found in Google
    req = urllib2.Request(url)
    response = urllib2.urlopen(req)
    the_page = response.read()
    soup = BeautifulSoup(the_page, 'lxml')
    find values look for
    extract information into a specific format
    write out tables
\end{lstlisting}

\subsubsection{Generate Welcome Page Task Design}
\begin{lstlisting}
{HTML for webpage layout}
{HTML for welcome image with a link}
if click_image == true
    link to Query Select Page
Else 
    no action
\end{lstlisting}

\subsubsection{Generate Query Select Page Task Design}
\begin{lstlisting}
{HTML for webpage layout}
{HTML for navigation bar}
{HTML for header with welcome image and introduction}
{HTML to display queries}
{HTML for provide users with input or options for each query}
if click_option == true 
    link to SQL Query
    query PresidentsElectionsDB
    get results
    link to Query Result Page
else 
    no action
if type_valid_input == true
    link to SQL Query
    query PresidentsElectionsDB
    get results
    link to Query Result Page
else
    return " The input is not valid."   
\end{lstlisting}

\subsubsection{Generate SQL Query Task Design}
\begin{lstlisting}
If query == Election_Year_Query
    SELECT c.year, p1.name, state, party, vote, votePercent, poll
    FROM Candidates c, Person p
    WHERE c.year = year_input and c.PID = p1.PID
Else if query == Given_Person_Query
    SELECT year, p.name,c.result
    FROM Person p, Candidates c
    WHERE p.name like name_input and c.PID = p.PID   
Else if query  == non-contiguous re-elect times query
    select p.name
    from Candidates c, Person p,
    (select p2.year as year1, p1.year as year2, p1.PID
    from presidents p1, presidents p2
    where p1.PID = p2.PID
    and p1.year - p2.year > 4) as temp
    where c.PID = temp.PID
    and c. year > year1
    and c.year < year2
    and result = 'fail'
    and c.PID = p.PID
Else if query == Swing_Candidates_Query
    SELECT c.year, p.name, party, result
    FROM Candidates c, Person p,
    (SELECT distinct c1.PID 
    FROM Candidates c1, Candidates c2
    WHERE c1.PID = c2.PID
    and c1.party != c2.party) as temp
    WHERE c.PID = temp.PID
    and c.PID = p.PID
Else if query == Party_Historical_Query
    SELECT year, vote, votePercent, result
    FROM Candidates 
    WHERE party = party_input
Else if query == Given_State_Query
    SELECT p.name
    FROM test.candidates c, test.person p
    WHERE c.pid = p.pid AND c. result = 'success' AND p.state = state_input
Else if query == Almost_President _Query
    create table x
    SELECT max(c.poll) as maxpoll,c.year
    FROM test.candidates c#, test.person p
    WHERE c.year > 1932
    GROUP by c.year;
    SELECT p.name
    FROM x, test.candidates c, test.Person p
    WHERE x.year = c.year and x.maxpoll = c.poll and c.result = 'fail' and c.PID = p.PID;
Else if query == Successfully_Re-election_Query
    CREATE TABLE presidents
    SELECT c.PID, p.name, c.year
    FROM test.candidates c, test.person p
    where c.PID = p.PID AND c.result = 'success';
    SELECT distinct p1.name
    FROM presidents p1, presidents p2
    WHERE p1.name = p2.name AND p1.year = p2.year-4;
Else if query == Dominate_Party
    SELECT count(c.PID), party
    FROM test.candidates c
    WHERE c.result = 'success' AND c.party != ''
    GROUP BY c.party
    ORDER BY count(c.PID) desc
\end{lstlisting}

\subsubsection{Generate Result Page Task Design}
\begin{lstlisting}
{HTML for webpage layout}
Generate table
    create table for results
    populate with results from Presidential Elections Database
    display the result able
If click_back == true
    link to Query Select Page  
\end{lstlisting}

\end{document}
